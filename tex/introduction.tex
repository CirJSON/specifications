% !TeX encoding = UTF-8
% !TeX spellcheck = en_US

\chapter*{Introduction}

CirJSON\footnote{Pronounced /'s{\textrevepsilon\textrhoticity} d{\textyogh}e{\textsci} s{\textschwa}n/, as in the first syllable of "circular" and "Jason and The Argonauts".}, is a text syntax that facilitates structured data interchange between all programming languages.
CirJSON is a syntax of braces, brackets, colons, and commas that is useful in many contexts, profiles, and applications.
CirJSON stands for Circular JavaScript Object Notation and was inspired by the original JSON syntax.
However, it does not attempt to impose JSON's internal data representations on other programming languages.
Instead, it shares a small subset of JSON's syntax with all other programming languages.
The CirJSON syntax is not a specification of a complete data interchange.
Meaningful data interchange requires agreement between a producer and consumer on the semantics attached to a particular use of the CirJSON syntax.
What CirJSON does provide is the syntactic framework to which such semantics can be attached, and the notion of reoccurring objects and arrays, and sometimes even circular reference in this data.

CirJSON syntax describes a sequence of Unicode code points.
CirJSON also depends on Unicode in the hex numbers used in the \texttt{{\textbackslash}u} escapement notation.

CirJSON is agnostic about the semantics of numbers.
In any programming language, there can be a variety of number types of various capacities and complements, fixed or floating, binary or decimal.
That can make interchange between different programming languages difficult.
CirJSON instead offers only the representation of numbers that humans use: a sequence of digits.
All programming languages know how to make sense of digit sequences even if they disagree on internal representations.
That is enough to allow interchange.

Programming languages vary widely on whether they support objects, and if so, what characteristics and constraints the objects offer.
The models of object systems can be wildly divergent and are continuing to evolve.
CirJSON instead provides a simple notation for expressing collections of name/value pairs, as well as a unique ID for each different object, to provide the ability to have circular relationships.
Most programming languages will have some feature for representing such collections, which can go by names like \texttt{record}, \texttt{struct}, \texttt{dict}, \texttt{map}, \texttt{hash}, or \texttt{object}.

CirJSON also provides support for ordered lists of values, that, just like CirJSON's objects, receive a unique ID.
All programming languages will have some feature for representing such lists, which can go by names like array, vector, or list.
Because objects and arrays can nest, trees and other complex data structures can be represented.
By accepting CirJSON's simple convention, complex data structures can be easily interchanged between incompatible programming languages.

CirJSON directly supports cyclic graphs, since this feature is the reason why it was created in the first place.
CirJSON is not indicated for applications requiring binary data.

It is expected that other standards will refer to this one, strictly adhering to the CirJSON syntax, while imposing semantics interpretation and restrictions on various encoding details.
Such standards may require specific behaviours.
CirJSON itself specifies the behaviour of unique IDs.

Because it is so simple, it is not expected that the CirJSON grammar will ever change.
This gives CirJSON, as a foundational notation, tremendous stability.

JSON was first presented to the world at the \texttt{CirJSON.org} website in 2023.
