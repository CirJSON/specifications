% !TeX encoding = UTF-8
% !TeX spellcheck = en_US

\section{Objects}

An object structure is represented as a pair of curly bracket tokens surrounding zero or more name/value pairs.
A name is a string.
A single colon token follows each name, separating the name from the value.
A single comma token separates a value from a following name.

After the opening curly bracket, there must be the name \texttt{"\_{\_}cirJsonId\_\_"}/\textit{ID} pair.

The CirJSON syntax has only one restriction on the strings used as names.
Except for defining the ID at the start of the object, the string \texttt{"\_{\_}cirJsonId\_\_"} cannot be used as a name.
Other than that, it does not require that name strings be unique, and does not assign any significance to the ordering of name/value pairs.
These are all semantic considerations that may be defined by CirJSON processors or in specifications defining specific uses of CirJSON for data interchange.

\begin{figure}[!htbp]
	\centering
	\tikzset{
		bas/.style={draw, fill=couleurback, thick, align=flush center, line width=2pt, text height=8pt},
		tok/.style={bas, rounded corners=8pt, text width=8pt},
		val/.style={bas, text width=6em},
		lit/.style={bas, rounded corners=8pt},
	}
	\begin{tikzpicture}[line width=2pt]
		\matrix[row sep=2em, column sep=1.5em] {
			\node{\textbf{\textit{object}}}; & ~ & ~ & ~ & ~ & ~ & ~ & ~ & ~ \\
			\node(start){~}; & \node[tok](op){\textbf{\texttt{\{}}}; & ~ & \node[lit](ik){\textbf{\texttt{"\_{\_}cirJsonId\_\_"}}}; & \node[tok](ic){\textbf{\texttt{:}}}; & \node[val](id){\textbf{\textit{ID}}}; & \node(he){~}; & \node[tok](cp){\textbf{\texttt{\}}}}; & \node(end){~}; \\
			\node(hc){~}; & \node[tok](fc){\textbf{\texttt{,}}}; & \node(hn){~}; & \node[val](nam){\textbf{\textit{string}}}; & \node[tok](pc){\textbf{\texttt{:}}}; & \node[val](val){\textbf{\textit{value}}}; & \node(hv){~}; & ~ & ~ \\
			~ & ~ & ~ & ~ & \node[tok](rc){\textbf{\texttt{,}}}; & ~ & ~ & ~ & ~ \\
		};

		\draw[|-] (start) -- (op.west);
		\draw[-] (op.east) -- (ik.west);
		\draw[-] (ik.east) -- (ic.west);
		\draw[-] (ic.east) -- (id.west);
		\draw[-] (id.east) -- (cp.west);
		\draw[-|] (cp.east) -- (end);

		\draw[-] (fc.east) -- (nam.west);
		\draw[-] (nam.east) -- (pc.west);
		\draw[-] (pc.east) -- (val.west);

		\linkReverseS{id}{he}{hc}{fc}

		\begin{scope}[x=1em,y=1em]
			\newdimen\xendOne
			\newdimen\xendTwo
			\newdimen\yend
			\path (he.east);
			\pgfgetlastxy{\xendOne}{\yend}
			\path (cp.west);
			\pgfgetlastxy{\xendTwo}{\yend}
			\newdimen\xstart
			\newdimen\ystart
			\path (val.east);
			\pgfgetlastxy{\xstart}{\ystart}
			\coordinate (1) at ({(\xendOne+\xendTwo)/2},\ystart);
			\coordinate (2) at ({(\xendOne+\xendTwo)/2},\yend);
			\draw[-,rounded corners] (val.east)--(1)--(2)--(cp.west);
		\end{scope}

		\linkHalfLoopRight{val}{he}{rc}
		\linkHalfLoopLeft{nam}{hn}{rc}
	\end{tikzpicture}
	\caption{object}
	\label{f:objs}
\end{figure}

% \begin{tikzpicture}[->,shorten >=1pt,auto,node distance=3cm,semithick]
%  	 \tikzstyle{every state}=[draw=black,text=black]
%
%  	 \node[initial by arrow,state,initial text=] (1) {1};
% 	 \node[state] (3) [right of=1] {3};
% 	 \node[accepting,state] (4) [below of=1] {4};
% 	 \node[state] (2) [below of=4] {2};
% 	 \node[state] (5) [right of=2] {5};
% 	 \node[state] (6) [right of=4] {6};
% 	 \node[state] (7) [below of=2] {7};
%
% 	 \path
% 	 (1) edge node {$a$} (4)
% 	 (1) edge node {$b$} (3)
% 	 (2) edge node {$a$} (7)
% 	 (2) edge node {$b$} (5)
% 	 (3) edge node {$a$} (4)
% 	 (3) edge node {$b$} (6)
% 	 (4) edge node {$a$} (2)
% 	 (4) edge node {$b$} (5)
% 	 (5) edge node {$a$} (7)
% 	 (5) edge node {$b$} (6)
% 	 (6) edge node {$a$} (4)
% 	 (6) edge [loop right] node {$b$} (6)
% 	 (7) edge [loop left] node {$a,b$} (7)
% 	 ;
%	
% \end{tikzpicture}
