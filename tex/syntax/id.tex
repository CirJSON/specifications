% !TeX encoding = UTF-8
% !TeX spellcheck = en_US

\section{ID}

An ID is a \textit{string} that is not empty.

\begin{figure}[!htp]
	\centering
	\tikzset{
		bas/.style={draw, fill=couleurback, thick, align=flush center, line width=2pt},
		tok/.style={bas, rounded corners=8pt, text width=8pt, text height=8pt},
		val/.style={bas},
	}
	\begin{tikzpicture}[line width=2pt]
		\matrix[row sep=1em, column sep=1em](mat) {
			\node{\textbf{\textit{IDs}}}; & ~ & ~ & ~ & \node(ab){~}; & ~ & ~ & ~ & ~ \\
			\node(start){~}; & \node[tok](bs){\texttt{"}}; & \node(as){~}; & ~ & ~ & ~ & \node(be){~}; & \node[tok](es){\texttt{"}}; & \node(end){~}; \\[1em]
			~ & ~ & ~ & \node[tok](se){\texttt{\textbackslash}}; & \node[tok](qm){\texttt{"}}; & ~ & ~ & ~ & ~ \\
			~ & ~ & ~ & ~ & \node[tok](rs){\texttt{\textbackslash}}; & ~ & ~ & ~ & ~ \\
			~ & ~ & ~ & ~ & \node[tok](so){\texttt{/}}; & ~ & ~ & ~ & ~ \\
			~ & ~ & ~ & ~ & \node[tok](ba){\texttt{b}}; & ~ & ~ & ~ & ~ \\
			~ & ~ & ~ & ~ & \node[tok](ff){\texttt{f}}; & ~ & ~ & ~ & ~ \\
			~ & ~ & ~ & ~ & \node[tok](lf){\texttt{n}}; & ~ & ~ & ~ & ~ \\
			~ & ~ & ~ & ~ & \node[tok](cr){\texttt{r}}; & ~ & ~ & ~ & ~ \\
			~ & ~ & ~ & ~ & \node[tok](ct){\texttt{t}}; & ~ & ~ & ~ & ~ \\
			~ & ~ & ~ & ~ & \node[tok](sh){\texttt{u}}; & \node[val](hd){4 hexadecimal digits}; & ~ & ~ & ~ \\
		};
		
		\node[val](cp) at ($(as)!0.5!(be)$) {\textbf{\textit{Any code point except \texttt{"} or \texttt{\textbackslash}}} \\ \textbf{\textit{or control character}}};
		
		\draw[|-] (start.east) -- (bs.west);
		\draw[-] (bs.east) -- (cp.west);
		\draw[-] (cp.east) -- (es.west);
		\draw[-|] (es.east) -- (end.east);
		\draw[-] (sh.east) -- (hd.west);
		
	\begin{scope}[x=1em,y=1em]
			\newdimen\useless
			\newdimen\xStart
			\newdimen\yStart
			\newdimen\xEnd
			\newdimen\yMid
			
			\path (as);
			\pgfgetlastxy{\xStart}{\yStart}
			\path (ab);
			\pgfgetlastxy{\useless}{\yMid}
			\path (be);
			\pgfgetlastxy{\xEnd}{\useless}
			
			\coordinate (1) at (\xStart,\yStart);
			\coordinate (2) at (\xStart,\yMid);
			
			\coordinate (3) at (\xEnd,\yMid);
			\coordinate (4) at (\xEnd,\yStart);
			
			\draw[-,rounded corners] (cp.west)--(1)--(2)--(3)--(4)--(cp.east);
		\end{scope}
		
		\multiLinksLeft{se}{qm,rs,so,ba,ff,lf,cr,ct,sh}
		\begin{scope}[x=1em,y=1em]
			\newdimen\useless
			\newdimen\xend
			\newdimen\yend
			\path (be.west);
			\pgfgetlastxy{\xend}{\yend}
			\newdimen\xstart
			\newdimen\ystart
			\path (hd.east);
			\pgfgetlastxy{\xstart}{\useless}
			\foreach \i in {qm,rs,so,ba,ff,lf,cr,ct,hd} {
				\path (\i.east);
				\pgfgetlastxy{\useless}{\ystart}
				\coordinate (1) at ({(\xend+\xstart)/2},\ystart);
				\coordinate (2) at ({(\xend+\xstart)/2},\yend);
				\ifdim\ystart=\yend
				\draw[-] (\i.east)--(be.west);
				\else
				\draw[-,rounded corners] (\i.east)--(1)--(2)--(es.west);
				\fi
			}
		\end{scope}
		
		\begin{scope}[x=1em,y=1em]
			\newdimen\xStart
			\newdimen\yStart
			\newdimen\xEnd
			\newdimen\yEnd
			
			\path (as.east);
			\pgfgetlastxy{\xStart}{\yStart}
			\path (se.west);
			\pgfgetlastxy{\xEnd}{\yEnd}
			
			\coordinate (1) at ({(\xStart+\xEnd)/2},\yStart);
			\coordinate (2) at ({(\xStart+\xEnd)/2},\yEnd);
			
			\draw[-,rounded corners] (as.east)--(1)--(2)--(se.west);
		\end{scope}
		
	\end{tikzpicture}
	\caption{ID}
	\label{f:ids}
\end{figure}
