% !TeX encoding = UTF-8
% !TeX spellcheck = en_US

\section{Numbers}

A number is a sequence of decimal digits with no superfluous leading zero.
It may have a preceding minus sign \texttt{–} (U+002D).
It may have a fractional part prefixed by a decimal point \texttt{.} (U+002E).
It may have an exponent, prefixed by \texttt{e} (U+0065) or \texttt{E} (U+0045) and optionally \texttt{+} (U+002B) or \texttt{–} (U+002D).
The digits are the code points U+0030 through U+0039.

Numeric values that cannot be represented as sequences of digits (such as \texttt{Infinity} and \texttt{NaN}) are not permitted.

\begin{figure}[!htbp]
	\centering
	\tikzset{
		bas/.style={draw, fill=couleurback, thick, align=flush center, line width=2pt},
		tok/.style={bas, rounded corners=8pt, text width=8pt, text height=8pt},
		val/.style={bas},
	}
	\begin{tikzpicture}[line width=2pt]
		\matrix[row sep=1em, column sep=0.5em] {
			\node{\textbf{\textit{number}}}; & ~ & ~ & ~ & ~ & ~ & \node(up){~}; & ~ & ~ & ~ & ~ \\
			\node(start){~}; & ~ & \node(bd){~}; & \node[tok](zer){0}; & ~ & \node[tok](dot){\texttt{.}}; & \node[val](dfl){\textit{digit}}; & ~ & ~ & ~ & \node(end){~}; \\
			~ & \node[tok](neg){\texttt{-}}; & ~ & \node[val](fdi){\textit{digit} \\ \textit{1-9}}; & ~ & ~ & ~ & ~ & ~ & ~ & ~ \\
		};
		
		\draw[|-] (start) -- (zer.west);
		\draw[-] (zer.east) -- (dot.west);
		\draw[-] (dot.east) -- (dfl.west);
		\draw[-|] (dfl.east) -- (end);
		
		\multiLinksRight{start}{neg}
		\multiLinks{neg}{bd}
	\end{tikzpicture}
	\caption{number}
	\label{f:nums}
\end{figure}