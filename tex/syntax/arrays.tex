% !TeX encoding = UTF-8
% !TeX spellcheck = en_US

\section{Arrays}

An array structure is a pair of square bracket tokens surrounding its ID and zero or more \textit{values}.
The \textit{values} and the ID are separated by commas.
The first value is the array's ID.
The CirJSON syntax does not define any specific meaning to the ordering of the following \textit{values}.
However, the CirJSON array structure is often used in situations where there is some semantics to the ordering.

\begin{figure}[!htbp]
	\centering
	\tikzset{
		bas/.style={draw, fill=couleurback, thick, align=flush center, line width=2pt, text height=8pt},
		tok/.style={bas, rounded corners=8pt, text width=8pt},
		val/.style={bas, text width=6em},
		lit/.style={bas, rounded corners=8pt},
	}
	\begin{tikzpicture}[line width=2pt]
		\matrix[row sep=2em, column sep=2em] {
			\node{\textbf{\textit{array}}}; & ~ & ~ & ~ & \node(up){~}; & ~ & ~ & ~ \\
			\node(start){~}; & \node[tok](op){\textbf{\texttt{[}}}; & \node[val](id){\textbf{\textit{ID}}}; & \node[tok](ic){\textbf{\texttt{,}}}; & \node[val](val){\textbf{\textit{value}}}; & \node(bp){~}; & \node[tok](cp){\textbf{\texttt{]}}}; & \node(end){~}; \\
			~ & ~ & ~ & ~ & \node[tok](ac){,}; & ~ & ~ & ~ \\
		};

		\draw[|-] (start) -- (op.west);
		\draw[-] (op.east) -- (id.west);
		\draw[-] (id.east) -- (ic.west);
		\draw[-] (ic.east) -- (val.west);
		\draw[-] (val.east) -- (cp.west);
		\draw[-|] (cp.east) -- (end);

		\linkHalfLoopLeft{val}{ic}{ac}
		\linkHalfLoopRight{val}{bp}{ac}

		\begin{scope}[x=1em,y=1em]
			\newdimen\useless
			\newdimen\xStaOne
			\newdimen\xStaTwo
			\newdimen\xEndOne
			\newdimen\xEndTwo
			\newdimen\ySta
			\newdimen\yEnd
			\path (id.east);
			\pgfgetlastxy{\xStaOne}{\ySta}
			\path (ic.west);
			\pgfgetlastxy{\xStaTwo}{\useless}
			\path (bp.east);
			\pgfgetlastxy{\xEndOne}{\useless}
			\path (cp.west);
			\pgfgetlastxy{\xEndTwo}{\useless}
			\path (up.east);
			\pgfgetlastxy{\useless}{\yEnd}
			\coordinate (1) at ({(\xStaOne+\xStaTwo)/2},\ySta);
			\coordinate (2) at ({(\xStaOne+\xStaTwo)/2},\yEnd);
			\coordinate (3) at ({(\xEndOne+\xEndTwo)/2},\yEnd);
			\coordinate (4) at ({(\xEndOne+\xEndTwo)/2},\ySta);
			\draw[-,rounded corners] (id.east)--(1)--(2)--(3)--(4)--(cp.west);
		\end{scope}
	\end{tikzpicture}
	\caption{array}
	\label{f:arrs}
\end{figure}
