% !TeX encoding = UTF-8
% !TeX spellcheck = en_US

\documentclass[12pt,twoside,titlepage]{report}

\usepackage{graphicx}
\usepackage[utf8]{inputenc}
\usepackage[T1]{fontenc}
\usepackage{tikz}
\usepackage{hyperref}
\usepackage{nameref}
\usepackage{geometry}
\usepackage{titlesec}
\usepackage{fancyhdr}
\usepackage{ifthen}
\usepackage{inconsolata}
\usepackage[documentfont=DoulosSIL]{tipauni}
\usepackage{lmodern}
\usepackage{float}
\usepackage{placeins}

\usetikzlibrary{arrows.meta}
\usetikzlibrary{positioning}
\usetikzlibrary{shapes}

\hypersetup{colorlinks,
	citecolor=black,
	filecolor=black,
	linkcolor=black,
	urlcolor=black}

% Definitions des dimensions de la page (package geometry)
\geometry{letterpaper,%
	centering,%
	hmargin={2.5cm,2.5cm},%
	vmargin={2.5cm,2.5cm},%
	heightrounded,%
	includehead}

\fancyhf{} % clear all header and footers
\renewcommand{\headrulewidth}{0pt} % remove the header rule
\fancyfoot[RE,LO]{\thepage} % Left side on Even pages; Right side on Odd pages
\pagestyle{fancy}

\renewcommand{\chaptername}{}
\titleformat{\chapter}[hang] 
{\normalfont\huge\bfseries}{\chaptertitlename\ \thechapter}{1em}{}

\renewcommand{\baselinestretch}{1.5}

\newfontfamily{\doulos}{DoulosSIL}

\newcommand{\multiLinksRight}[2]{
	\begin{scope}[x=1em,y=1em]
		\newdimen\xstart
		\newdimen\ystart
		\path (#1.east);
		\pgfgetlastxy{\xstart}{\ystart}
		\foreach \i in {#2} {
			\newdimen\xend
			\newdimen\yend
			\path (\i.west);
			\pgfgetlastxy{\xend}{\yend}
			\coordinate (1) at ({(\xend+\xstart)/2},\ystart);
			\coordinate (2) at ({(\xend+\xstart)/2},\yend);
			\ifdim\ystart=\yend
			\draw[|-] (#1.east)--(\i.west);
			\else
			\draw[|-,rounded corners] (#1.east)--(1)--(2)--(\i.west);
			\fi
		}
	\end{scope}
}

\newcommand{\multiLinksBase}[3]{
\begin{scope}[x=1em,y=1em]
	\newdimen\xend
	\newdimen\yend
	\path (#2.west);
	\pgfgetlastxy{\xend}{\yend}
	\foreach \i in {#1} {
		\newdimen\xstart
		\newdimen\ystart
		\path (\i.east);
		\pgfgetlastxy{\xstart}{\ystart}
		\coordinate (1) at ({(\xend+\xstart)/2},\ystart);
		\coordinate (2) at ({(\xend+\xstart)/2},\yend);
		\ifdim\ystart=\yend
		\draw[#3] (\i.east)--(#2.west);
		\else
		\draw[#3,rounded corners] (\i.east)--(1)--(2)--(#2.west);
		\fi
	}
\end{scope}
}

\newcommand{\multiLinks}[2]{\multiLinksBase{#1}{#2}{-}}
\newcommand{\multiLinksEnd}[2]{\multiLinksBase{#1}{#2}{-|}}

\newcommand{\linkReverseS}[4]{
\begin{scope}[x=1em,y=1em]
	\newdimen\xStartOne
	\newdimen\yStartOne
	\path (#1.east);
	\pgfgetlastxy{\xStartOne}{\yStartOne}
	\newdimen\xStartTwo
	\newdimen\yStartTwo
	\path (#2.west);
	\pgfgetlastxy{\xStartTwo}{\yStartTwo}
	\newdimen\xEndOne
	\newdimen\yEndOne
	\path (#3.east);
	\pgfgetlastxy{\xEndOne}{\yEndOne}
	\newdimen\xEndTwo
	\newdimen\yEndTwo
	\path (#4.west);
	\pgfgetlastxy{\xEndTwo}{\yEndTwo}

	\coordinate (1) at ({(\xStartOne+\xStartTwo)/2},{(\yStartOne+\yStartTwo)/2});
	\coordinate (2) at ({(\xStartOne+\xStartTwo)/2},{(\yStartOne+\yEndTwo)/2});
	\coordinate (3) at ({(\xEndOne+\xEndTwo)/2},{(\yStartOne+\yEndTwo)/2});
	\coordinate (4) at ({(\xEndOne+\xEndTwo)/2},{(\yEndOne+\yEndTwo)/2});
	\draw[-,rounded corners] (#1.east)--(1)--(2)--(3)--(4)--(#4.west);
\end{scope}
}

\newcommand{\linkHalfLoopRight}[3]{
\begin{scope}[x=1em,y=1em]
	\newdimen\xStartOne
	\newdimen\yStart
	\newdimen\xStartTwo
	\newdimen\yEnd
	\path (#3.east);
	\pgfgetlastxy{\xStartOne}{\yEnd}
	\path (#2.west);
	\pgfgetlastxy{\xStartTwo}{\yStart}
	\path (#1.east);
	\pgfgetlastxy{\xStartOne}{\yStart}
	
	\coordinate (1) at ({(\xStartOne+\xStartTwo)/2},\yStart);
	\coordinate (2) at ({(\xStartOne+\xStartTwo)/2},\yEnd);
	\draw[-,rounded corners] (#1.east)--(1)--(2)--(#3.east);
\end{scope}
}

\newcommand{\linkHalfLoopLeft}[3]{
\begin{scope}[x=1em,y=1em]
	\newdimen\xStartOne
	\newdimen\yStart
	\newdimen\xStartTwo
	\newdimen\yEnd
	\path (#3.west);
	\pgfgetlastxy{\xStartOne}{\yEnd}
	\path (#2.east);
	\pgfgetlastxy{\xStartTwo}{\yStart}
	\path (#1.west);
	\pgfgetlastxy{\xStartOne}{\yStart}
	
	\coordinate (1) at ({(\xStartOne+\xStartTwo)/2},\yStart);
	\coordinate (2) at ({(\xStartOne+\xStartTwo)/2},\yEnd);
	\draw[-,rounded corners] (#1.west)--(1)--(2)--(#3.west);
\end{scope}
}

\title{CirJSON}
\author{The CirJSON Data Interchange Syntax}

\begin{document}
	% !TeX encoding = UTF-8
% !TeX spellcheck = en_US

\pagenumbering{roman}
\maketitle
\cleardoublepage
\tableofcontents
\cleardoublepage
\addcontentsline{toc}{section}{\listfigurename}
\listoffigures
\cleardoublepage
\pagenumbering{arabic}

	% !TeX encoding = UTF-8
% !TeX spellcheck = en_US

\chapter*{Introduction}

CirJSON\footnote{Pronounced {\doulos/'s{\textrevepsilon\textrhoticity} d{\textyogh}e{\textsci} s{\textschwa}n/}, as in the first syllable of "circular" and "Jason and The Argonauts".}, is a text syntax that facilitates structured data interchange between all programming languages.
CirJSON is a syntax of braces, brackets, colons, and commas that is useful in many contexts, profiles, and applications.
CirJSON stands for Circular JavaScript Object Notation and was inspired by the original JSON syntax.
However, it does not attempt to impose JSON's internal data representations on other programming languages.
Instead, it shares a subset of JSON's syntax with all other programming languages.
The CirJSON syntax is not a specification of a complete data interchange.
Meaningful data interchange requires agreement between a producer and consumer on the semantics attached to a particular use of the CirJSON syntax.
What CirJSON does provide is the syntactic framework to which such semantics can be attached, and the notion of reoccurring objects and arrays, and sometimes even circular reference in this data.

CirJSON syntax describes a sequence of Unicode code points.
CirJSON also depends on Unicode in the hex numbers used in the \texttt{{\textbackslash}u} escapement notation.

CirJSON is agnostic about the semantics of numbers.
In any programming language, there can be a variety of number types of various capacities and complements, fixed or floating, binary or decimal.
That can make interchange between different programming languages difficult.
CirJSON instead offers only the representation of numbers that humans use: a sequence of digits.
All programming languages know how to make sense of digit sequences even if they disagree on internal representations.
That is enough to allow interchange.

Programming languages vary widely on whether they support objects, and if so, what characteristics and constraints the objects offer.
The models of object systems can be wildly divergent and are continuing to evolve.
CirJSON instead provides a simple notation for expressing collections of name/value pairs, as well as a unique ID for each different object, to provide the ability to have circular relationships.
Most programming languages will have some feature for representing such collections, which can go by names like \texttt{record}, \texttt{struct}, \texttt{dict}, \texttt{map}, \texttt{hash}, or \texttt{object}.

CirJSON also provides support for ordered lists of values, that, just like CirJSON's objects, receive a unique ID.
All programming languages will have some feature for representing such lists, which can go by names like array, vector, or list.
Because objects and arrays can nest, trees and other complex data structures can be represented.
By accepting CirJSON's simple convention, complex data structures can be easily interchanged between incompatible programming languages.

CirJSON directly supports cyclic graphs, since this feature is the reason why it was created in the first place.
CirJSON is not indicated for applications requiring binary data.

It is expected that other standards will refer to this one, strictly adhering to the CirJSON syntax, while imposing semantics interpretation and restrictions on various encoding details.
Such standards may require specific behaviours.
CirJSON itself specifies the behaviour of unique IDs.

Because it is so simple, it is not expected that the CirJSON grammar will ever change.
This gives CirJSON, as a foundational notation, tremendous stability.

JSON was first presented to the world at the \texttt{CirJSON.org} website in 2023.

	% !TeX encoding = UTF-8
% !TeX spellcheck = en_US

\chapter{The CirJSON Data Interchange Syntax}

% !TeX encoding = UTF-8
% !TeX spellcheck = en_US

\section{Scope}

CirJSON is a lightweight, text-based, language-independent syntax for defining data interchange formats.
It was derived from the JSON data interchange format, and is programming language independent.
CirJSON defines a small set of structuring rules for the portable representation of structured data.

The goal of this specification is only to define the syntax of valid CirJSON texts.
Its intent is not to provide any semantics or interpretation of text conforming to that syntax.
It also intentionally does not define how a valid CirJSON text might be internalized into the data structures of a programming language.
There are many possible semantics that could be applied to the CirJSON syntax and many ways that a CirJSON text can be processed or
mapped by a programming language.
Meaningful interchange of information using CirJSON requires agreement among the involved parties on the specific semantics to be applied.
Defining specific semantic interpretations of CirJSON is potentially a topic for other specifications.

% !TeX encoding = UTF-8
% !TeX spellcheck = en_US

\section{Conformance}

A conforming CirJSON text is a sequence of Unicode code points that strictly conforms to the CirJSON grammar defined by this specification.
A conforming processor of CirJSON texts should not accept any inputs that are not conforming CirJSON texts.
A conforming processor may impose semantic restrictions that limit the set of conforming CirJSON texts that it will process.

% !TeX encoding = UTF-8
% !TeX spellcheck = en_US

\section{CirJSON Text}

A CirJSON text is a sequence of tokens formed from Unicode code points that conforms to the CirJSON value grammar.
The set of tokens includes six structural tokens, strings, numbers, and three literal name tokens.

The six structural tokens:

\begin{table*}[htp]
	\begin{tabular}{lll}
		\texttt{[} & U+005B & left square bracket \\
		\texttt{\{} & U+007B & left curly bracket \\
		\texttt{]} & U+005D & right square bracket \\
		\texttt{\}} & U+007D & right curly bracket \\
		\texttt{:} & U+003A & colon \\
		\texttt{,} & U+002C & comma \\
	\end{tabular}
\end{table*}

These are the three literal name tokens:

\begin{table*}[htp]
	\begin{tabular}{llllll}
		\texttt{true} & U+0074 & U+0072 & U+0075 & U+0065 & ~ \\
		\texttt{false} & U+0066 & U+0061 & U+006C & U+0073 & U+0065 \\
		\texttt{null} & U+006E & U+0075 & U+006C & U+006C & ~ \\
	\end{tabular}
\end{table*}

Insignificant whitespace is allowed before or after any token.
Whitespace is any sequence of one or more of the following code points:

\FloatBarrier

\begin{table*}[!htbp]
	\begin{tabular}{ll}
		character tabulation & U+0009 \\
		line feed & U+000A \\
		carriage return & U+000D \\
		space & U+0020 \\
		line separator & U+2028 \\
		paragraph separator & U+2029 \\
	\end{tabular}
\end{table*}

\FloatBarrier

Whitespace is not allowed within any token, except that space is allowed in strings.

% !TeX encoding = UTF-8
% !TeX spellcheck = en_US

\section{CirJSON Values}

A CirJSON value can be an \textit{object}, \textit{array}, \textit{number}, \textit{string}, \textit{ID}, \texttt{true}, \texttt{false}, or \texttt{null}.

\begin{figure}[!htbp]
	\centering
	\tikzset{
		val/.style={draw, fill=couleurback, thick, text width=6em, align=flush center,line width=2pt},
		lit/.style={val, rounded corners=8pt},
	}
	\begin{tikzpicture}[line width=2pt]
		\matrix[row sep=1em, column sep=12em] {
			\node{\textbf{\textit{value}}}; & ~ & ~ \\
			\node(start){~}; & \node[val](obj){\textbf{\textit{object}}}; & \node(end){~}; \\
			~ & \node[val](arr){\textbf{\textit{array}}}; & ~ \\
			~ & \node[val](num){\textbf{\textit{number}}}; & ~ \\
			~ & \node[val](str){\textbf{\textit{string}}}; & ~ \\
			~ & \node[val](ids){\textbf{\textit{ID}}}; & ~ \\
			~ & \node[lit](tr){\textbf{\texttt{true}}}; & ~ \\
			~ & \node[lit](fl){\textbf{\texttt{false}}}; & ~ \\
			~ & \node[lit](nl){\textbf{\texttt{null}}}; & ~ \\
		};
		
		\multiLinksRight{start}{obj,arr,num,str,ids,tr,fl,nl}
		\multiLinksEnd{obj,arr,num,str,ids,tr,fl,nl}{end}
	\end{tikzpicture}
	\caption{CirJSON Values}
	\label{f:vals}
\end{figure}


\end{document}